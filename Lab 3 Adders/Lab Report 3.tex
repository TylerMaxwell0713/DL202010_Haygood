\documentclass[]{report}
%==========================================================
%=========== Document Setup  ==============================

% Formatting defined by class file
%\documentclass[11pt]{article}
% ---- Document formatting ----
\usepackage[margin=1in]{geometry}
% Narrower margins
\usepackage{booktabs}                    
% Nice formatting of tables
\usepackage{graphicx}                    
% Ability to include graphics%
\setlength\parindent{0pt}    
% Do not indent first line of paragraphs
\usepackage[parfill]{parskip}    
% Line space b/w paragraphs
% parfill option prevents last line of pgrph from being fully justified
% Parskip package adds too much space around titles, fix with this
 \RequirePackage{titlesec}
 \titlespacing\section{0pt}{8pt plus 4pt minus 2pt}{3pt plus 2pt minus 2pt}
 \titlespacing\subsection{0pt}{4pt plus 4pt minus 2pt}{-2pt plus 2pt minus 2pt}
 \titlespacing\subsubsection{0pt}{2pt plus 4pt minus 2pt}{-6pt plus 2pt minus 2pt}
% ---- Hyperlinks ----
 \usepackage[colorlinks=true,urlcolor=blue]{hyperref}  
% For URL's. Automatically links internal references.
% ---- Code listings ----
 \usepackage{listings}                          % Nice code layout and inclusion
 \usepackage[usenames,dvipsnames]{xcolor} % Colors (needs to be defined before using colors)
 % Define custom colors for listings
 \definecolor{listinggray}{gray}{0.98}          % Listings background color
 \definecolor{rulegray}{gray}{0.7}              % Listings rule/frame color
 % Style for Verilog
 \lstdefinestyle{Verilog}{
  language=Verilog,                        % Verilog
  backgroundcolor=\color{listinggray},     % light gray background
  rulecolor=\color{blue},                  % blue frame lines
  frame=tb, % lines above & below
  linewidth=\columnwidth,                  % set line width
  basicstyle=\small\ttfamily,   % basic font style that is used for the code
  breaklines=true,                         % allow breaking across columns/pages
  tabsize=3, % set tab size
  commentstyle=\color{gray},     % comments in italic
  stringstyle=\upshape,                    % strings are printed in normalfont
  showspaces=false,                        % don't underscore spaces
 }
% How to use: \Verilog[listing_options]{file}
\newcommand{\Verilog}[2][]{%
\lstinputlisting[style=Verilog,#1]{#2}
}
\begin{document}

\title{ELC 2137 Lab 3: Adders}
\author{Tyler Haygood, Ivan Rios}
\maketitle

\section*{Summary}
The Purpose of this lab was to familiarize the class with the basics of Adders.
We were introduced to Half, Full, and Two-Bit adders in the PreLab with their
individual truth tables, and then furthered our understanding by building them in the lab
using XOR, and AND gates.



\section*{Q\&A}
\begin{enumerate}
\item Which gates could weuse  for  combining  the  carry  bits? \newline
We could have used an OR gate to combine the carry bits however, we used the XOR gate for convenience.

\end{enumerate}

\section*{Results}


\begin{figure}[ht]
\centering

\includegraphics{CircuitDemo.PNG}
\caption{Circuit Demonstration Page}
\label{fig:Page One of Demo Page}

\end{figure}

\begin{figure}[ht]
\centering
\includegraphics{HalfAdder.PNG}
\caption{Half Adder Circuit}
\label{fig:Half Adder}
\end{figure}

\begin{figure}[ht]
\centering
\includegraphics{FullAdder.PNG}
\caption{Full Adder Circuit}
\label{fig:Full Adder}
\end{figure}

\begin{figure}[ht]
\centering
\includegraphics{TwoBitAdder.PNG}
\caption{Two Bit Adder Circuit}
\label{fig:Two bit Adder}
\end{figure}



\end{document}